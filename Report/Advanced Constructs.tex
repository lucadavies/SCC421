\documentclass{article}
\usepackage[utf8]{inputenc}
\usepackage[a4paper, margin=1in]{geometry}
\usepackage[font={small,it}]{caption}
\usepackage{graphicx}
\usepackage{xcolor}
\usepackage{pdfpages}
\usepackage{natbib}
\usepackage{graphicx}
\usepackage{float}
\usepackage{url}
\usepackage{nameref}
\usepackage{pdflscape}
\usepackage{listings}
\usepackage{xparse}
\usepackage{hyperref}

\graphicspath{{../Figures/}}
\bibliographystyle{agsm}
\NewDocumentCommand{\codeword}{v}{%
\texttt{{#1}}%
}
\lstset{language=C,keywordstyle={\bfseries}}
\hypersetup{
    colorlinks,
    citecolor=black,
    filecolor=black,
    linkcolor=black,
    urlcolor=black
}

\title{\textbf{Examining the Usage of Advanced Programming Constructs}}
\author{
Luca Davies \\ M.Sci. (Hons.) Computer Science (with Industrial Experience)}
\date{4th June 2021}

\begin{document}
\maketitle

\newpage
\section*{Declaration}
    I certify that the material contained in this dissertation is my own work and does not contain unreferenced or unacknowledged material. I also warrant that the above statement applies to the implementation of the project and all associated documentation. Regarding the electronically submitted work, I consent to this being stored electronically and copied for assessment purposes, including the School’s use of plagiarism detection systems in order to check the integrity of assessed work. \\
    I agree to my dissertation being placed in the public domain, with my name explicitly included as the author of the work. \\
    
    \noindent
    Name: Luca Davies\\
    Date: 01/06/2021
\newpage
\section*{Abstract}
    As time has gone on, programming languages have become more developed and more advanced syntax constructs have been added alongside many "quality of life" type structures to make the writing code not only more efficient but also easier for developers. This study examines seven of these advanced constructs, drawing together both data from surveys taken by professional developers and data from analysis of four widely known, active, open source code bases. Using a short but broad survey, data was collected on developers' usage of these constructs. This showed that developers primary concern when choosing how to write code lies with keeping it clear and simple, but that their definition of these things come more from their own personal preference, not style guides. Analysis of open source code was carried using a combination of simple regular expressions and hand analysis. A commits were found that enforce use of some constructs and replaced the plain syntax in nearly all occurrences. Examining of the the intermediate languages of compiled C\# and Java also showed some marked differences in the produced code dependant on the syntax used. Constructs that conventinally might be thought of as interchangable do have a visible effect on intermediate code.
    \newline
    \newline
    [Is there any working docs?]
    \newline
    %Special Thanks?
\newpage
\tableofcontents

\newpage
\section{Introduction}
    \subsection{Overview}
    \label{sec:overview}
        Code style has long been a topic of heated debated within the field of computer science since even the very early days. You will rarely ever get a single consensus in room when asking ``How should code blocks be indented?'', or worse still ``What is the best brace style?''. These kinds of physical layout questions have been around as long as the languages they exist within, but as programming languages have become more complex and afford programmers greater ease to express their logic, this discussion of best style has only broadened. Languages include many `syntactic sugar' constructs - patterns and operators that allow the simplification of previously more complex-looking or verbose statements, commonly thought of as a friendlier or otherwise specialised piece of syntax. The constructs that will be examined in this study are listed below:

        \begin{itemize}
            \item Ternary/in-line If Statements ( \codeword{a ? b : c} )
            \item Null-coalesce ( \codeword{a ?? b} ) and Null-conditional Operators ( \codeword{a?.b} / \codeword{a?[x]b})
            \item Lambda Expressions and Anonymous Functions ( \codeword{(a) => { b; }} )
            \item Additional constructs:
            \begin{itemize}
                \item Foreach Loops ( \codeword{foreach (a in b)} / \codeword{for ( a : b )})
                \item Unary Increment Operators (\codeword{a++}, \codeword{b--})
                \item Compound Assignment Operators (\codeword{a += 2},  \codeword{b -= 2}, \codeword{c *= 2}, \codeword{d /= 2}, etc...)
            \end{itemize}
        \end{itemize}
        
        \emph{Detailed descriptions of these constructs are given in Section \ref{subsec:constructs}}\\

        These particular constructs were selected as they represent a selection of commonly used syntax of varying complexity. They are also present in many modern programming languages found in use in industry today. These constructs represent a sample of all such constructs that could be categorised similarly.
        \\
        Prior to this project, I was working on a large C\# codebase with a nearly 20 year history. Throughout that time, the notion of `ideal' style had changed not only within the company the codebase belonged to, but in the wider software industry too. This led to wide mix of `plain' syntax together with many of the newer syntax constructs. For example, lambdas were only added to C\# around seven or eight \emph{years} after the first iteration of the codebase had been written. Some files featured a complete lack of any of these more advanced syntax constructs, some had many, some had some very difficult to read examples of their use, where perhaps it may have been better to have used more traditional syntax. For example, one code snippet took the form:\newline

        \codeword{ myVar = conditionA ? (doSomething(paramA)}\newline
        \indent\indent\codeword{: conditionB ? doSomethingElse(paramB))}\newline
        \indent\indent\codeword{: (conditionC ? doADifferentThing(paramC)}\newline
        \indent\indent\codeword{: doAnotherThing(paramD))}\newline
        

        It is this mix-and-match of the use of these constructs that spurred the idea of this project. Looking at code, there are clearly times when traditional syntax is overly verbose and takes up more space on the screen than necessary, but it also became abundantly clear that to swap out \emph{all} plain syntax for these more compact styles would be a big mistake, making many bits of code much harder to read.

        So where does the line lie between when to, and when to not use these constructs? To answer this, we must consider why we would use either one or the other at all. Often they are used to make code more compact - as software development has grown as an industry, there are certain `set' statements that it can generally be assumed that everyone understands, and that can be abstracted a little to make them neater or smaller. Other times it may be what has been laid out in a company style guide that everyone should use, or perhaps purely the habits and practices of an individual. However, regardless of the exact reason or motivation behind every use of these constructs, it is important to remember that no tool is unable to be misused - overuse of these types of constructs or use of them in inappropriate places, can make code more complicated and harder to read than it would have been otherwise. It is equally important to value the `plain' or verbose syntax as much as the compact variants, both having very unique and distinct advantages and disadvantages.
        \\

        This project began with an aim to make recommendations on how to best use the constructs in question, but as research continued and it was found just how little clear guidance is available, and how much conflict and indecision is present, it has become a more exploratory study in the use and usage of theses constructs. To do so, we focus on C\#, JavaScript, and Java as a representative sample of popular imperative programming languages.

        They may be used for multiple reasons, such as to make code simpler, or clearer. Other times, it may be style defined either by a programmer themselves, or the house rules of their organisation.  In this paper, we will study the above constructs and how they are used, with a view to making recommendations about they are best used, and where they are best avoided.
    \subsection{Motivation}
        The primary drive for this study stems from my experience during my industrial placement (SCC.419). As briefly mentioned above, the codebase I was working on was extensive and developed by numerous developers over the course of the last two decades or so. There were times that code was either made easier or harder not by its flow, but by the way it was written. The first instance that came to my attention was a ternary if-statement used that was so long that it needed to be split across four lines (similar to that shown in Section \ref{sec:overview}), had a second and third ternary in the true \emph{and} false branch of the enclosing ternary, with lengthy expressions being evaluated from there. In this instance, it can be assumed that it would have been significantly clearer to use a regular if-statement. Despite this the existence of ternary if-statements is still useful, but it brings to question where these constructs and other similar constructs should best be used.

        There also seems to have been relatively little research into languages features such as these, either in terms of subjective readability or objective performance. Some studies I have come across in research even make explicit note of the fact their definition of `good style' is subjective and opinion-based, albeit from that of perceived experts. Using experts in a field is always a good sign, but if there really is very little guidance to any level of programmer, then it stands to reason that much of what even experts think is good style, could be from personal experience alone, or at least somewhat unacademic.
        
        It is hoped that by formally examining these constructs in an academic context that the produced research will a launch pad or grounding for more thorough examination across the software development industry as a whole. Research into this topic has the potential to positively influence industry-wide best practices, both directly with developers themselves, and wtih authors of standards and programming guides. Essentially, this research is of direct relation to code style in general, which is equally important to anyone writing code in any high-level programming languages and even some lower-level languages as well.
    \subsection{Aims \& Objectives}
        The aims of this report are as follows:
        \begin{enumerate}
            \item To understand the usage of seven advanced programming constructs (as defined in Sections \ref{sec:overview} and \ref{subsec:constructs}).
            \item To study what reasons developers have to, or to not, use these constructs.
            \item To research how these constructs are actually used in practice.
            \item To examine the effects of using these constructs has on compiled code.
        \end{enumerate}


        These aims will be achieved via the following objectives:
        \begin{enumerate}
            \item Collect data from professional developers about when and how they think it is appropriate to use these constructs via a survey.
            \item Investigate what guidance is available to developers on how to use these constructs by examining available style documents provided by developers and online.
            \item Examine the recommendations made by static code analysis tools on specifically written sample code files to act as `control' tests (containing back-to-back examples of plain syntax and syntax using the constructs).
            \item Analyse four open source repositories to assess how these constructs are used in practice both by hand and using static code analysis tools.
            \item Study the intermediate language (where applicable) produced by plain syntax contrasted with syntax using the constructs.
            \item If time and resources permit: conduct software profiling and performance tests of code that both do and do not use the constructs to discover if they have a measurable effect on performance.
        \end{enumerate}

    \subsection{Methods}
        A multifaceted approached was taken to achieve the above aims, incorporating multiple methods. First, a survey was carried out, to gather both qualitative and quantitative data direct from developers about how they use each of the constructs being examined - this helps address aim one. A survey was chosen as it provides a comparatively time- and resource-efficient method to gather answers to base questions from a large sample size.
        Quantitative data was collected, with regard to aim two, by employing static code analysis tools to analyse four open source GitHub repositories and sample code files (that contain examples of both the constructs and their `plain' equivalents), in three programming languages. Static analysis tools were employed as they most likely represent actual tools developers may use to check their code, and thus the rules and recommendations that will be levelled against their code in a real scenario.
        Aims one and two were also supported by the analysis of the commit history of the four repositories for changes that include or exclude the use of the constructs in question. Lastly, aim three will be met via a culmination of all methods used, and some simple numerical analysis.

    \subsection{Threats to Validity}
        Due to time constraints and limits of contacts, the survey was only able to encompass one group of programmers, all from within one company. Although the technology stack the developers in question may have been exposed to consisted primarily of C\# and JavaScript, this potentially means participants may have been answering questions with little knowledge on Java. The results of the survey are also difficult to generalise, as there are many factors which may affect responses that are limited to this one group of developers.

        The survey was distributed widely within ISS, giving approximately 60 potential respondents. Of those, there are 44 recorded responses  (~70\% response rate), 23 of which are completed, valid responses (~40\% valid responses).
    \subsection{Report Structure}
        The remainder of this report will first detail further background research directly and indirectly related to this topic, before describing the methods and design of the study, followed by the results and information gained from the study, discussion on these results, before final conclusions.
\newpage
\section{Background}
\label{sec:background}
    At face-value, much of the grounding of this study may be considered conventional wisdom. That is, the constructs are simply used anywhere and everywhere according to when we as programmers deem it appropriate, without much prior thought or formal procedure. This once would have been the case for structural or physical style to our code, the way we lay our code out and arrange it around whitespace. Countless style guides have concerned themselves with this for many years, and there's still much debate around which brace style is the best in C-like languages, or how many spaces should be used to indent code blocks. It hasn't been until much more recent times, as software development houses have grown, and the computing industry as a whole has ballooned into the incredibly important sector it is today, that we have started to think more about the meta-style of our programming. Not how it's physically laid out, but how we layout our logic and design patterns. Somewhere between these two schools of thought lies the area this study is focussed on.

    It does not appear, however, that a great deal has been written and published around how to best use many of the programming constructs that many programmers will use tens, if not hundreds, of times per day. What follows is a summary of much of the existing works that cover partially on wholly the same topics.
    \subsection{Early Style Metrics}
        As far back as the 1980s, \cite{berryMeekingStyle} proposed the Berry-Meeking Style Metric as a measure of good style and readability of code. This used style analysers adapted specifically to C to produce a metric denoting the `lucidity' of the code. The metric was derived from 11 different characteristics about a program: module length, identifier length, percentage of comment lines, amount of indentation, percentage of blank lines, line length, spaces within lines, percentage of constants, reserved words used, included files and number of gotos used. Zooming out from this level of metric-based scoring, \cite{paradigmForStyleResearch} collects a number of studies of together and notes how application of these types of style metrics not only lack correlation with error proneness, but also has unpredictable effects on complexity metrics. They go further to say that style is an `intuitive and elusive' concept, that is `highly individualistic' and simultaneously easy to see, but hard to define. Many differing style guides are either too general or subjective, contradictory between each other and with no guidance and how to manage these conflicts. They go on to liken the use programming style to ``the `perception and judgement' a writer exercises in selecting `from equally correct expressions, the one the best suited to his material, audience and intention'{}''. This is to say that the nuance of writing in English prose is present too in the style of code.

    \subsection{Automatic Feedback Systems for Students}
        In more recent times, it is again reinforced that some simple metrics like program length cannot solely be used to measure the quality of code style. \cite{autoStyleFeedbackAtScale}, when showing a range of students' responses to a programming task (the shortest of which is unlikely to be rated the easiest to read), state that length is not a sole indicator of good style, and moreover that ``excessive terseness is often worse than that of verbosity''. We still don't have any \emph{definite} way to judge `good' style, with \cite{scaleDrivenHints} noting that their study on providing style hints to new programmers is limited by the fact the given `good' style is subjective and opinionated.

    \subsection{Popular \& Official Guides}
        One of the defining style guides for Java, Effective Java \citep{effectiveJava} makes many assertions about how to use certain features of the Java language, but most of these do not reach the level of granularity focussed on in this study. The reference to anything close is that Java programmers should ``prefer for-each loops over traditional for loops''. Lambdas feature in their own chapter, but only with reference to \emph{how} to use them, not \emph{when}.

        It is reasonable to think that official style guides published by programming language developers would be cover-all for all sorts of style, from the extraneous and finnicky to most basic syntax, but in fact, Microsoft's own style guide for C\# is somewhat short, and it's only relevant mention for this study being a note that lambdas are `shorter' and may thus be good substitutions for the more verbose regular methods that would be needed otherwise.

        Two prevailing style guides for JavaScript have come into prominence the last few years, both with certain small conflicts and differences from each other. These are the Google style JavaScript style guide \citep{googleJSStyle}, and the Airbnb JavaScript style guide \citep{airbnbJSStyle}. The Google guide advocate for use of \codeword{for-of} loops over all other types of \codeword{for} loop in JavaScript, allowing \codeword{for-in} loops exceptionally for dict-style objects. The Airbnb guide, conversely to Google, requires use of \emph{iterators} over \codeword{for-in} or \codeword{for-of}, with no reference to standard for loops. It also makes one very relevant assertion: that ternaries should ``not be nested and generally be single line expressions''. Perhaps controversially, the Airbnb guide also rejects the use of the unary operators, \codeword{++} and \codeword{--}. This is primarily due to the fact that these statement cause automatic semicolon insertion and can cause silent errors. However, it is also reasoned that the more expressive \codeword{+=} and \codeword{-=} operators allow code to be easier to read and less error-prone.
    \subsection{Constructs}
        \label{subsec:constructs}
        This section will give a more detailed description of each of the constructs to be examined. The constructs were selected for a number of reasons - firstly, they are all replaceable with simpler syntax (lambdas being a occasional exception, see Section \ref{subsubsec:lambdas}), conventional wisdom would dictate that some are rather commonly used (compound assignment / unary operators) and conversely that some are used much less (null-coalesce / null-conditional). Initial thoughts of some of the constructs came about from seeing them during my work for SCC.419. The constructs in question are detailed below:
        \begin{itemize}
            \item Ternary/in-line If Statements
                \begin{itemize}
                    \item Reduces the common \codeword{if} \dots \codeword{else} \dots pattern into a single line of the form \codeword{a ? b : c}, where \codeword{a} is a conditional expression, \codeword{b} is the body of the \codeword{if} and \codeword{c} is the body of the \codeword{else}. While it may be possible in some languages to have multiple statement in the `b' and `c' sections of a ternary, but standard syntax only permits a single statement.
                \end{itemize}
            \item Lambdas and Anonymous Functions
                \begin{itemize}
                    \item A lambda expression is any expression involving a function used as an argument, just as in Lambda Calculus. Though this is not required, lambdas most often take the form of anonymous (unnamed) functions defined in-line. \citep{javaLambdas}.
                \end{itemize}
            \item Null-coalescing Operator
                \begin{itemize}
                    \item Used to provide a default value in place of a null value. Replaces a null-check (using an if statement) with a single assignment of the form \newline\codeword{a = possiblyNullValue  ?? valueIfNull} where \codeword{a} is of a nullable type. \citep{cs5Spec}.
                \end{itemize}
            \item Null-conditional Operator
                \begin{itemize}
                    \item Used to access a member or element of its operand \emph{if and only if} that operand evaluates to non-null. Otherwise, it returns null. \citep{cs6Spec}. This allows a member/element access operation to be carried out without a null-check on the base operand, only on the evaluated value.
                \end{itemize}
             \item Foreach Loops
                \begin{itemize}
                    \item Used to simplify for loops for the purposes of iterating over an abstract data structure, usually a type of some form of collection of like objects; those that can be indexed into to retrieve an element. For example, in C\#, a foreach loop may be used to iterate over the elements of an instance of any class that implements the \codeword{IEnumerable} interface. The same applies to Java with its \codeword{Iterable} interface. In C\#, if \codeword{numList} is of type \codeword{List<int>}:
                    \\
                    \codeword{for (int i = 0; i < numList.Count; i++)}\\
                    \{\\
                    \hspace*{1cm}\codeword{total += numList[i];}\\
                    \}
                    \\\\
                    Can be written as:\\
                    \codeword{foreach (int num in numList)}\\
                    \{\\
                    \hspace*{1cm}\codeword{total += num;}\\
                    \}
                \end{itemize}
            \item Unary Increment Operator
                \begin{itemize}
                    \item Used to increment or decrement an integer variable by one. Often seen in \codeword{for} loops to increment the index variable.
                    \item For example:\\
                    \codeword{a = a + 1;}\\\\
                    Can be written as:\\
                    \codeword{a++;}
                \end{itemize}
            \item Compound Assignment Operators
                \begin{itemize}
                    \item Used to effectively ``append'' to a variable's value, compressing three references to variables/literals into two. In most supporting languages, a compound assignment operator exists for all basic arithmetic operators, logic operators and some others (such as in string concatenation).
                    \item For example:\\
                    \codeword{a = a * 2;}\\\\
                    Can be written as:\\
                    \codeword{a *= 2}
                \end{itemize}
        \end{itemize}
    
\begin{landscape}
    \begin{figure}[htbp]
        \centering
        \vspace{2in}
        \includegraphics[width=1.5\textheight]{timeline}
        \caption{Timeline of the addition of advanced constructs to three modern languages.}
        \label{fig:timeline}
    \end{figure}
\end{landscape}
\newpage
\section{Study Design}
    \subsection{Survey}
    \label{subsec:survey}
        One of the facets of this study was to capture and analyse the perceptions and thoughts of a number of professional software developers with regard to the constructs being examined.
        
        This was carried out using a survey, constructed following guidelines laid down by both published literature (\cite{goodSurveys1}, \cite{goodSurveys2}) as well as a small amount of grey literature provided by higher education bodies. This allowed better questions to be written, that ensured they were no longer than necessary, used no double negatives, were not leading questions, only attempted to address one issue, and were not too vague. The design was also guided by these sources to include at least one open-ended question to allow participants to put forward any exceptional or unexpected information that could prove highly valuable to the study. In the literature, rightfully, notable mention is made at all stages to the use of appropriate moral and ethical frameworks. In this instance, very little in the way of any problematic material is examined, and no issues arose during the ethics apporval process in the University. The most personal question asked in the survey is how many years participants have been programming for - no other questions of a remotely personal nature are asked.

        Using the Qualtrics platform also allowed the usability of the survey to be examined by Qualtrics' inbuilt system. This advised the design of the survey to be as easy to use as possible, increasing the likelihood of respondents completing their submissions. The only element of the survey to go against this was the use of matrix table questions, which are not particularly friendly to mobile users. Despite this, they were used as they provided the best and most concise way to display the literal matrix of options available due to the presence of multiple constructs per question. Figure \ref{fig:matrixTable} shows one such question. The full survey is available in Appendix \ref{apx:survey}
        
        \begin{figure}[htbp]
            \centering
            \includegraphics[width=0.8\textwidth]{matrixTable}
            \caption{A matrix table question used in the survey.}
            \label{fig:matrixTable}
        \end{figure}

        \subsubsection{Research Questions}
        \label{subsubsec:surveyQuestions}
            Prior to designing, it was established that the survey must be able to provide insight that may help to answer key research questions. As software development and programming are taught in a multitude of different ways, we can make no assumptions about how familiar participants would be with any of the constructs. To address this, the base questions had to answered: \emph{What} constructs do they know of? \emph{When} do they use the constructs? \emph{Why} do they use the constructs? From this, some additional, more speculative questoins could be asked. The following questions were formulated to best gather as much information as possible:
            \begin{enumerate}
                \item Are the participants \emph{aware} of the constructs?
                \item Do the participants \emph{use} the constructs?
                \item Are participants \emph{encouraged to use} the constructs?
                \item Do the participants think these constructs are better for:
                \begin{enumerate}
                    \item brevity?
                    \item clarity?
                    \item code efficiency/performance?
                \end{enumerate}  
                \item Do the participants think that the constructs are better used in some languages than in others?
                \item Do participants rewrite code either \emph{to use} or \emph{to not use} the constructs?
            \end{enumerate}
            
        \subsubsection{Participants}
            A convenience sample of participating developers was recruited from within the Information Systems Services department (ISS) within Lancaster University by means of a blanket email sent out to all ISS teams and filtered down via regular communications channels within the department. This gathered 40 responses in total, 23 of which completed the survey and may be considered valid responses.
            
            Due to the distribution method, there is no guarantee that all respondents are software developers by profession, however, all participants reported a minimum of one year of programming experience, nearly half falling into the 1-4 years of experience category, with three participants having between 10 and 14 years experience and the remaining 8 having over 20 years of experience. This statistic was the only personal data collected and does not identify participants.
        \subsubsection{Questions}
            The survey was created using Qualtrics, as provided by Lancaster University, and consisted of a maximum of 16 questions and a minimum of 2 questions (depending on certain answers).
            Participants were first asked to indicate which of the constructs they were aware of so that only questions and options relating to those constructs that they were aware of were shown to them. For each construct they were aware of, participants were asked to answer a series of questions designed to address the questions listed in Section \ref{subsubsec:surveyQuestions}, focussing on: frequency of use, reasons to and to not use each construct, perception of effect on code performance, and perception of industry-wide usage per-language.
            
            Lastly, participants were asked if they had any further comments, which invited some very interesting points that will be discussed alongside the rest of the results in Section \ref{subsec:results}.
    \subsection{Static Code Analysis}
        Static code analysis is concerned with checking programs and code for errors without the need to actually execute it. That is, a static code analysis tool will read the source code (or the compiled intermediary language in some cases), construct some form of abstract model of the program and then run a series of tests (often pattern-matching in nature) to detect well-known possible pit-falls, bugs, and problems \cite{staticCodeAnalysis}.

        These tools will often make recommendations on how source code may be improved in numerous way beyond explicit bugs and problems, such as stylistic improvement that make use of advanced syntax whatever given language is being analysed - advanced syntax akin to the constructs being examined here. Accordingly, static analysis tools were employed to investigate whether or not they made any recommendations for or against any of the constructs in question. SonarQube and Semgrep were selected for this purpose as they both support all three languages being subject to analysis, they are both under recent, active development, and because they both have free variants. Both tools were run against sample control code files and large open source repositories in the earlier mentioned three languages: C\#, JavaScript, and Java.
            
        \subsubsection{Sample Code Files}
        \label{subsubsec:sampleFile}
            The sample code files were created to act as a baseline or control test to see if SonarQube or Semgrep made any note at all about the bare use and/or inclusion of any of the constructs.

            Each file contains a bare (non-contextualised) example of each of the constructs placed directly alongside their `simple syntax' functionally equivalent counterparts. Each file is valid syntax for its language, and all of them may be executed, though none of them have a true entry point or way to actually \emph{run} the code within itself - the functions and methods are present purely to \emph{be present} so that they may be analysed by SonarQube and Semgrep. Though there are differences between the three languages, the form of the sample code was kept as similar as possible to preserve their purpose as control samples, without adding deliberately unusual syntax that may contaminate the output of SonarQube and Semgrep.
        \subsubsection{Open Source Repositories}
            Four large, open source code repositories were selected from GitHub to be analysed for this study, one each in C\# and JavaScript, and two in Java. The selection process was thought to be a random selection from the first page of the most popular GitHub repos for each language, however, it was discovered late in the process the list from which the repos were selected was actually the `daily trending' listing.  Fortunately, the repos that were selected, while potentially not as representative as initially thought, are still very well known, very active repositories maintained by a diverse range of developers.
    \subsection{Git Commit Analysis}
        To supplement the static analysis of the repositories, manual, by-hand analysis of the commits made to these repositories was also undertaken.

        The goal of this branch of the study was to understand if and how standards are being maintained within the repositories within reference to the constructs and to general style-keeping. This was carried out by cloning the repository and examining the commit history using \codeword{git log}. A list of keywords were run against the commit history using the \codeword{--grep=<pattern>} option to \codeword{git log}. The keywords used are listed below:
        
        \begin{center}
            \begin{tabular}{ | l | l | }
                \hline
                \textbf{General} & style \\
                \hline
                \textbf{Ternary} & ternary, conditional, \codeword{?:}, elvis \\
                \hline
                \textbf{Null coalesce/conditional} & null, coalesce, conditional, \codeword{??}, \codeword{?.}, \codeword{?[} \\
                \hline
                \textbf{Lambda} & lambda, arrow, \codeword{=>}, \codeword{->} \\
                \hline
                \textbf{For each} & foreach, for each \\
                \hline
                \textbf{Unary Operators} & unary, increment, decrement, ++, -{}- \\
                \hline
                \textbf{Compound Operators} & compound, assign, \codeword{+=}, \codeword{-=}, \codeword{*=}, \codeword{/=}, \codeword{|=}, \codeword{&=} \\
                \hline
            \end{tabular}
        \end{center}

        Some of these keywords generated cross-over between those meant to highlight changes around a particular construct, but this was unimportant as these keywords were only used to create a list of  `interesting' commits. Commits were deemed interesting if the commit message and/or description contained reference to any of the keywords in such a way that it seem plausible that a change was made regarding any relevant construct. For each commit in the resultant list, the changes were scrutinised closely to pick out changes made that: added, removed, altered, or commented on use of any of the constructs. Each type of change was documented and counted per commit.

    \subsection{Intermediate Language Comparison}
        The final part of this study was brought about during the progress of the preceding work. Analysis of the survey results raised some interesting points and potential ideas around what may make certain constructs more or less suited to use generally and/or in any given place in code.

        C\# and Java, while both are complied languages, they are compiled to an intermediate language which is then interpreted by a virtual machine for execution. This means that we may examine the intermediate language (IL) to understand if there are any clear differences between the plain and advanced forms of syntax and constructs. This would be a somewhat crude way to go about attempting to gauge the difference in performance or efficiency, but merely examining and commenting no the difference in C\# and Java's respective ILs can prove to valuable supporting material to this study.
        
        The code files used to generate the IL code contained the same code as in the sample files mentioned in Section \ref{subsubsec:sampleFile} with one change: the sample code files have consecutive functions, one in plain syntax, one using an advance construct. To make analysis and comparison easier, the plain and advanced syntax functions were extracted into separate files so that there was a plain syntax \emph{file} and an advanced syntax \emph{file}. This allowed ease of comparison using git integrations with Visual Studio Code to perform a \codeword{git diff} to see the similarities and differences easily. 

        For C\#, the online tool SharpLab \footnote{https://sharplab.io/} was used to produce .NET Common Intermediate Language using the Microsoft's Roslyn  compiler. For Java, `JVM Bytecode Viewer' for Visual Studio Code \footnote{https://marketplace.visualstudio.com/items?itemName=mnxn.jvm-bytecode-viewer} was used to locally open the .class files produced by the JDK to view the code.

\newpage
\section{Results}
    \subsection{Survey}
    \label{subsec:results}
        As described in Section \ref{subsec:survey}, a survey was conducted with experienced professional software developers and programmers. This revealed interesting patterns of usage (or lack of usage) of the constructs, and some interesting perceptions that may prove useful in future research focussed on performance and use in the wider industry.

        The following sections will detail the results of the survey, including threats to the usefulness of the survey and the patterns discovered by the survey. All percentage values given have been rounded to 1 decimal place unless otherwise stated.
        \subsubsection{Awareness}
            Awareness of the constructs was gathered by a simple multiple choice question with a binary yes/no for each. Participants could indicate whether or not were familiar with all, some, or none of the constructs.
            \\\newline
            All participants were aware of unary operators and ternary/in-line if-statements, nearly all were aware of compound operators and lambdas, and most were aware of null-coalesce and null-conditional. Figure \ref{fig:awareness} shows these results in a graphic format.

            \begin{figure}[htbp]
                \centering
                \includegraphics[width=0.8\textwidth]{awareness}
                \caption{Percentage of participants aware of each construct}
                \label{fig:awareness}
            \end{figure}

            The most stand-out piece of information here is that the \emph{absolute} awareness of the null-coalesce and -conditional operators was markedly higher than might be expected. This can likely be attributed to the fact that participants were all from within ISS, where the  technology stack contains a large amount of C\#, in which these operators are often seen. If the survey were repeated over a more diverse and broad range of developers working in more varied technology stacks (for example, those focussed much more on Java, or C/C++), it is reasonable to assume that awareness of these operators would be less. This is a trend that is repeated elsewhere in the results set.
            It is also of note that not \emph{all} participants were aware of compound assignment operators, whereas all participants \emph{were} aware of ternaries - a ranking inverse to what was expected. However, the difference in the numbers is not very large, comprising no more than one or two responses. To draw any valid conclusion from this datum would require it being replicated in a greater sample set.
        \subsubsection{Frequency of Use}
            To assess how often participants use each of the constructs, they were asked specifically: \textit{In places where they can be used, how often do you use each of these constructs in your code?}, that is taken to mean:, ``for every chance you \emph{could} use the construct, how often do you?''
            \\\newline
            It was the extreme ends of the scales in this question that provided the most interesting results here. Firstly, only one data point was recorded for any construct under the `never' option, with just one respondent indicating that they never use the null-coalesce operator. This is perhaps reassuring given that it's generally accepted that most of these constructs have at least \emph{some} place to be used. Conversely, `for each' was the most selected construct in the `always' category, with 44\% of participants selecting this option, followed closely by compound assignment operators at 41\%.

            Across all constructs, the majority of responses fell into the `frequently' category, holding 39\% of the total selections, with the `sometimes' category at 27\%. This general erring toward the more frequent end of the scale holds not only for the collective dataset but also for each construct when examined individually, as may be seen in figure \ref{fig:freqUse}

            \begin{figure}[htbp]
                \centering
                \includegraphics[width=0.9\textwidth]{freqUse}
                \caption{Frequency with which participants said they used advanced constructs, grouped by construct. (n.b. figures normalised to percentages)}
                \label{fig:freqUse}
            \end{figure}

            As is visible in figure \ref{fig:freqUse}, the use of lambdas in moderation is the most agreed upon data point here, with just over 70\% of responses labelled `sometimes' for lambdas. This is likely explainable by a variety of reasons, not least is the fact that lambdas are very powerful and complex structures, much more so than the other constructs being studied, and as such, the number of situations where they \emph{can} be used likely greatly exceeds the number situations where they \emph{should} be used. This distinction around lambdas becomes more apparent in later questions.
            Of note, again, are the results for the null-coalesce and null-conditional operators. Although the numbers of participants familiar with these operators is fewer, those that were aware of them do not shy away from them, using them with comparatively similar frequency to the other constructs.
        \subsubsection{Reasons To Use}
            Participants were then asked to indicate \emph{why} they chose to use each given construct. The four options were presented:
            \begin{itemize}
                \item ``It is company style''
                \item ``It makes code clearer''
                \item ``It makes codes simpler''
                \item ``It makes code run faster''
            \end{itemize}
            One or more of the above options may have been selected, with an additional exclusive option: ``None of these'' present too. If this was selected, participants were prompted to give their own reason in the following question.
            \newline

            The trend here was extremely clear: the most common reasons for using any of the constructs was for clarity and simplicity. ``It makes code clearer'' and ``It makes code simpler'' took 39.4\% and 42.3\% of \emph{all} selections made respectively. The data for the selections totalled across each construct is shown in Figure \ref{fig:toUsePie}.

            Six participants selected `None of these' for at least one construct, two of whom selected this option for all constructs they were aware of and both instead said that they choose to use the constructs based on personal preference or habit. Another two selected this option for lambdas alone, one saying that they are aware of, but don't use lambdas, and the other highlighting a possible flaw in this study regarding lambdas. This participant noted that there are places where ``there is no rational alternative to using a lambda expression. It does not make the code simpler or clearer, it's just \textbf{necessary} to use a lambda.'' This brought to attention the fact that lambdas may be a more functional language feature, not fitting in the same same category as the other constructs here, which can mostly be categorised as `syntactic sugar' that do not have much actual difference on code functionality. This is discussed further in Section \ref{subsubsec:lambdas}.
            One participant also said that they would generally use `map' instead of a foreach loop. `Map' is presumably referring to the \codeword{map()} in JavaScript and similar that may be called on an array to perform a function on all elements of the array - the function is taken by \codeword{map()} as an argument, either via lambda or regular method or function. They claim using \codeword{map()} ``helps avoid `off by one' errors''.  Off-by-one-errors occur when iterating over an array or other list of \textit{m} through \textit{n} items and having the wrong index by one. For example, if you were to ask how many items there are (how many loop iterations are required) in a collection, the intuitive answer of \textit{m} - \textit{n} is incorrect, the correct answer being (\textit{m} - \textit{n}) + 1. This is illustrated using a specific type of off-by-one error, a \textit{fencepost error}, as illustrated in Figure \ref{fig:fencepost}.
            \begin{figure}[htbp]
                \centering
                \includegraphics[width=0.5\textwidth]{fencepost}
                \caption{If you have a fence with 10 posts, how many bounded sections exist between the posts? The intuitive answer is 10, but it is in fact 11. This is a specific off-by-one error.}
                \label{fig:fencepost}
            \end{figure}

            It is not clear what advantages a `map' function has over a foreach loop to avoid off-by-one errors. In fact, conventional wisdom dictates that foreach loops, by nature should, avoid these errors by virtue of the fact that they\emph{remove} the need for a programmer to calculate the necessary number of loop iterations or use an index at all. It is plausible to suggest that this response was given by mistake, instead meaning to refer to traditional for loops, rather than foreach loops.

            Figure \ref{fig:toUse} part repeats the data above, again showing that clarity and simplicity are the most common reasons. We can also see that the split is \emph{mostly} even between these two reasons, null-coalesce being a slight exception, with nearly double as many responses selecting simplicity over clarity. Null-conditional follows this pattern, but the difference is less pronounced.
            Other notable statistics are: no participants said that they used null-coalesce or -conditional due to company style, and the three constructs that had code run speed selected. Two participants selected this for lambdas, and one each both selected this for foreach loops and unary operators. These selections do not represent a large proportion of the total, but it is still of note that none of the other constructs had any selections regarding faster code execution.
            
            \begin{figure}[htbp]
                \centering
                \includegraphics[width=0.9\textwidth]{toUsePie}
                \caption{Reasons to use constructs, totalled across all constructs.}
                \label{fig:toUsePie}
            \end{figure}
            \begin{figure}[htbp]
                \centering
                \includegraphics[width=0.9\textwidth]{toUse}
                \caption{Reasons to use constructs grouped by construct. (n.b. Figures normalised to percentages per construct)}
                \label{fig:toUse}
            \end{figure}

        \subsubsection{Reasons To Not Use}
            To provide the other half of the story to the previous section, participants were also asked about the reasons they choose to \emph{not} use the constructs. Similar to before, the options presented were:
            \begin{itemize}
                \item ``It is company style''
                \item ``It would make code less clear''
                \item ``It would make code more complex''
                \item ``It would make code run slower''
            \end{itemize}
            As before, one or more of the above options may have been selected, with a ``None of these'' option that prompted participants to give their own reason(s) if selected.
            \newline

            Reasons of clarity and complexity were the most selected again, with 48.2\% of all selections attributing maintaining the clarity of code as a reason to not use a given construct, and complexity at 23.9\%. More than double the proportion of selections were present for ``None of these'' compared to the when asked for reason \emph{to} the constructs - here, 16.2\% of selections were placed here. The reasons provided by these respondents were quite varied but very interesting and are all worthy of discussion:

            \begin{itemize}
                \item Personal preference
                \begin{itemize}
                    \item Mentioned explicitly by three, and implicitly by one, of the ten individuals who gave their own reasons.
                    \item One noted that ``as long as code is well documented you can use any of them''.
                    \item One wrote that there is no company style for these constructs and that there is instead guidance on larger/less granular things like examples of a model class. This is contradicted by other responses, though it's plausible they are subject to different coding standards within ISS.
                    \item The third, the only of the ten that overlap with those that selected this option on the previous question. Simply stating that it is personal preference.
                    \item The final reason links the usage of compound and unary operators going against their preference to use a `functional style' and to `avoid mutability of variables'. This raises interesting questions about the use of functional programming ideas in non-functional paradigms.
                \end{itemize}
                \item `I [almost] always use this construct'
                \begin{itemize}
                    \item Two participants said they would always use unary increment operators.
                    \item One said they would always use compound operators.
                    \item Another said the only time there would not use null-coalesce/null-conditional is when null checking is not required at all.
                \end{itemize}
                \item Other/Unqualified
                \begin{itemize}
                    \item One participant stated that a for loop may be preferable to a foreach loop when the index variable is needed. They then go on to say that it also `depends how I'm feeling'.
                    \item Regarding lambdas and unary and compound operators, one participant said: ``sometimes these operators don't account for requirements such as type checking or formatting'' without any more description.
                    \item One participant claimed that foreach loops may introduce errors, but did not explain further.
                \end{itemize}
            \end{itemize}

            \begin{figure}[htbp]
                \centering
                \includegraphics[width=0.9\textwidth]{toNotUsePie}
                \caption{Reasons to not use constructs, totalled across all constructs.}
                \label{fig:toNotUsePie}
            \end{figure}
            \begin{figure}[htbp]
                \centering
                \includegraphics[width=0.9\textwidth]{toNotUse}
                \caption{Reasons to not use constructs grouped by construct. (n.b. Figures normalised to percentages per construct)}
                \label{fig:toNotUse}
            \end{figure}

            Only 4.5\% selected ``It would make code run slower'' - logically, this would make sense, factually too, as most of these constructs are plain syntactic differences that, with most modern compilers and interpreters, may produce the same bytecode and/or machine code. Lambdas and foreach loops are likely the only exceptions to this. The former has other potential implications beyond just how it looks on-screen, the latter (depending on language) may be a change from an array index to initialising an iterator and calling on that instead. This is reflected in the thoughts of participants in Figure \ref{fig:toNotUse}, showing that these two constructs were the only two to have received any selections pertaining to code run speed.

            The breakdown in Figure \ref{fig:toNotUse} also confirms that clarity was the most common reason selected, leadings by a wide margin for four of the constructs, but not for unary and compound operators (multiple reasons, described above) and for foreach loops (run speed was a concern as described immediately above).
            All constructs had a small number of responses claiming company style as reasons for not using them, but this is a relatively small number compared to the proportion that cited clarity or complexity for their reason.
        \subsubsection{Performance}
            As touched on previously, it was not expected with most of these constructs for there to be much of a bias with regard to performance (this is examined crudely in Section \ref{subsubsec:sharpLab}). Participants were asked to state what, if any, effect they thought each construct would have on the performance of code.
            \newline

            In line with expectation, a large majority, 61.2\%, of the selections here indicated no change to performance. The second largest group was ``Don't know'' with 22.4\% of selections. Both these statistics are shown in Figure \ref{fig:performance}b. It is likely that most high-level programmers (such as those using high-level languages like C\#, JavaScript, and Java) either do not routinely give much consideration to such fine-grained efficiency, or are not concerned with relatively small performance variations that come from using these constructs.

            The individual performance data also matches participants' selections from the previous question when grouped by construct: the majority for each is still ``No change'', but the stand out constructs that have a high number of selections of increased or decreased performance are lambda expressions and foreach loops. Lambdas had the smallest proportion of selections for ``No change'', and the largest for ``Increased performance'', meanwhile foreach had the second smallest proportion of ``No change'' and the largest for ``Decreased performance''. The possible decrease in performance using foreach was discussed in the previous section.

            \begin{figure}[htbp]
                \centering
                \includegraphics[width=0.8\textwidth]{performance}
                \caption{(a) Selections for performance, totalled across all constructs.\newline(b)Performance selections grouped by construct. (n.b. Figures normalised to percentages per construct)}
                \label{fig:performance}
            \end{figure}   
        \subsubsection{Usage In Wider Industry}
            The final set of questions asked participants about their perception of the prevalence of the constructs in the wider industry of software development, asking whether they thought that use of each construct was above, on par, or below average in any of the three languages when compared with use across the industry in general. Figure \ref{fig:industry} collates results across all constructs for this question set.

            \begin{figure}[htbp]
                \centering
                \includegraphics[width=0.9\textwidth]{industry.png}
                \caption{Responses on prevalence in wider industry compared to each language, displayed per construct. A typographical error meant that participants were asked about null-coalesce in Java: this should have been JavaScript, as Java has no null-coalesce operator. That portion of the results should be ignored. (n.b. Figures normalised to percentages and truncated scale, Null-coalesce and null-conditional operators are not present in Java, giving only two groupings)}
                \label{fig:industry}
            \end{figure}

            \begin{figure}[htbp]
                \centering
                \includegraphics[width=0.9\textwidth]{industry.png}
                \caption{The same as Figure \ref{fig:industry}, but excluding ``Don't know'' responses.}
                \label{fig:industryExcl}
            \end{figure}

            In general, responses were quite scattered, and the most clearly discernible patterns are centred on the ``Don't know'' responses.

            The most consistent pattern across the entire set of responses was that of uncertainty around usage of the constructs in Java. This will likely but due to the fact that ISS does not use Java in its tech stack anywhere. The most obvious case of this is for ternary if-statements, with 60.9\% of all responses for Java being ``Don't know''. Also, lambda expressions and foreach loops were perceived as being more common in C\# and JavaScript by a majority.

            The following breakdown summarises results for each construct, excluding ``Don't know'' responses from figures:
            
            \begin{itemize}
                \item In-line if / Ternary If Statements
                \begin{itemize}
                    \item the only construct that received a large number of selections in any language a notable number of ``Below average'' selections, the other selections were also high giving a mean value corresponding to just marginally \emph{above} average for C\#, marginally \emph{below} average for JavaScript and just above the midpoint between below average and average for Java.
                \end{itemize}
                \item Lambda Expressions
                \begin{itemize}
                    \item At least one of each selection, but with ``Above average'' outstanding for C\#, and particularly so for JavaScript. C\# and JavaScript have a mean corresponding to just midway between average and above average, and Java a little above average
                \end{itemize}
                \item Null-coalesce
                \begin{itemize}
                    \item The mean for C\# gives a value a little above average, while, as mentioned above, a typographic error means no data was collected for JavaScript and the Java data is unusable.
                \end{itemize}
                \item Null-conditional
                \begin{itemize}
                    \item Both trend toward average/above average for with C\# having a mean lying at the midpoint between average and above average and JavaScript lying a little above average.
                \end{itemize}
                \item Foreach Loops
                \begin{itemize}
                    \item Alongside lambdas, the only other construct strongly perceived as ``Above average''. The mean value for C\# and Java is strongly toward above average while JavaScript has a mean a little above average.
                \end{itemize}
                \item Unary Increment Operators
                \begin{itemize}
                    \item Means were above average for all languages: C\# midway between average and above average, JavaScript marginally above, and Java strongly above average.                
                \end{itemize}
                \item Compound Assignment Operators
                \begin{itemize}
                    \item Means were again above average for all languages, a little above average for all three.  
                \end{itemize}
            \end{itemize}

            \begin{figure}[htbp]
                \centering
                \includegraphics[width=0.9\textwidth]{industryPie.png}
                \caption{Responses on prevalence in wider industry compared to each language, totalled across all constructs and languages.}
                \label{fig:indsutryPie}
            \end{figure}

            In Section \ref{sec:conclusion}, the Conclusion, we will compare the figures in this section with the data collected in the analysis of real-world codebases to see if there is a meaningful correlation by which we may state that the participants surveyed are well aware of the prevalence of use of these constructs.

        \subsubsection{Free-form Comments}
            The final question asked participants to indicate any other thoughts and notes they had around the topic at hand. This yielded some very interesting comments.

            As has been mentioned already, most of these constructs come under the umbrella of `syntactic sugar', that which is meaningful only to a human developer, and function the same as their simpler counterparts at machine level. One participant noted this themselves, and further stated that these constructs should be used to save space and time, so long as code readability is not compromised. One more nuanced response gave much more detail, opening with a phrase that sums up the very issue of knowing when to use the constructs: ``Most of these questions [...] are more properly answered `As appropriate''', going on to say that runtime performance ought to be measured only in context, not in isolation. They apply this statement both across different scenarios in the same language and across different languages. They also highlight that in the present day ``we're waiting for [input/output] from the disk or network - the time spent executing each one of those constructs pales in comparison''. The same comment again reaffirms that ``everything depends on circumstances'', such as in safety-critical systems where code understandability and correctness are of ultimate importance. In such situations, it may be advisable to only use basic syntax and work around that as an axiom, but then again, it may not - proving the point that the context and situation may be one of the most important factors in usage of such advanced constructs, and may also go some way to explain why there seems to be little explicit guidance on this matter.

            Another comment stated again that the constructs being examined are syntactic sugar, however, this was written in a way that implied something to the effect ``they are just syntactic sugar and so they don't really matter''. This may not have been the case, but it would be further explanation as how uncertain the situation appears to be around when and how to use such constructs. The same comment separates lambdas out as not syntactic sugar, but a structure that permits more functional practices that can ``fundamentally change how your system is built and hence how it is complier optimised''. This is one of the pieces of information gathered during the process that has quite clearly marked out lambdas as not quite fitting into the same category as the rest of the constructs.

            Lastly, one participant notes that there are times when some of these constructs simply cannot be used, giving the example of writing entity framework expressions in C\# using LINQ. It is perhaps up for debate if LINQ counts as C\# on more than technicality, but it is a valid point that there are places where certain syntax cannot be written or where available syntax is otherwise limited. However, these situations are a form of edge-case, and are such an edge-case that, should standardised documents be written to define how to use the constructs, it could easily make exceptions for such situations.

        \subsubsection{Threats to Survey}
            The group of survey participants was comprised of a convenience sample of developers from with ISS. This means that the results may not generalise very well to a wider group of developers. The results may be generalised \emph{within} ISS but not outside. Further research and more significant surveying should be undertaken to verify if the results found within ISS may be applied to a wider populace of developers.

            Despite this, a survey still provided a very quick and  easy way to gather information from a sample size much larger than would be feasible using many other methods.
    \subsection{Static Code Analysis}
        []
        \subsubsection{Sample Code Files}
            Bugger all picked up, tools do not care
            \begin{itemize}
                \item Picked up basic \codeword{for} in place of JS \codeword{for-to} (not \codeword{for/each})
            \end{itemize}
        \subsubsection{Real-World Code Bases}
            []
        \subsubsection{Intermediate Language Comparison}
        \label{subsubsec:sharpLab}
            To aid in confirming or denying some of the claims made in the survey by some participants, and to provide more general information about the constructs, the intermediate language code produced C\# and Java compilers was compared for both their `plain' and `advanced' syntax.

            C\# was complied using SharpLab and the Roslyn compiler from March 3\textsuperscript{rd} 2021. Java was compiled locally with JDK version 14.0.1 and the .class files were read by the JVM Bytecode Viewer extension for Visual Studio Code.

            The comparison will be broken down by construct and the detailed per language:

            \begin{itemize}
                \item In-line / Ternary If-statement
                \begin{itemize}
                    \item \textbf{C\#} - Both versions are the same length, and both use two return statements (as is used in the plain source code). In the source code, the Boolean expression used is \codeword{temp < 20.0}: in the \emph{plain} IL, the logic for the condition is inverted, using a `branch if greater than or equal' instruction while the advanced IL uses a `branch if less than' instruction as would normally be assumed.
                    \item \textbf{Java} - Plain follows the source code one-to-one, using two return instructions at the end of each code branch, whereas advanced uses a \codeword{goto} instruction on the true branch to jump to a single return instruction used by both code branches. This means that on one code branch, one additional instruction will be executed when using a ternary if statement. In the same fashion is in C\#'s plain IL, both Java IL versions invert the Boolean logic and use a `branch if greater than or equal' instruction.
                \end{itemize}
                \item Lambda Expressions
                \begin{itemize}
                    \item \textbf{C\#} - The lambda in the advanced version is much more complex in IL than in source: a form of an entire class is created to hold the actual method that is defined in line in source. This class has a public and private constructor and the wanted method, generating 54 lines of code that an completely absent in the plain version. The method itself within this class is identical to that of the plain version. The actual code that \emph{calls} the lambda is only three instructions, compared to the plain version's six, however, this is also preceded by nine additional instructions that create and load an instance of the lambda's class. For such a small code snippet as is used here, it seems as though the potential overhead of a lambda may not necessary. In the case of reuse however, this could change. This also goes further still to showing that lambdas are a deeper construct than they had were initially given credit for in this study.
                    \item \textbf{Java} - Very similar to C\# again, the in-place calling of the lambda is shorter than the plain equivalent, but there is additional code elsewhere in the file, effectively defining the lambda method as if it were a regular method (which is again the same length of the plain method). The inclusion of a lambda also adds 22 entries to the constant pool to help find the lambda method and call it at runtime. In Java, this boilerplate amounts to 43 additional lines in unedited bytecode.
                \end{itemize}
                \item Null-coalesce (C\# only)
                \begin{itemize}
                    \item \textbf{C\#} - The plain version here is quite one-to-one with the source, declaring two \codeword{Person} variables and assigning them within code branches as expected. In contrast, the advanced version only uses the stack, and does not declare any local variables. This also has the side effect of requiring a bigger stack allocation to run the method, the advanced version requiring a size of eight to the plain version's one. This format is likely attributed to some degree of optimisation taking place by seeing that the variable in the condition is only null ever set to null in the method. This optimisation is potentially dependant on the context.
                \end{itemize}
                \item Null-conditional (C\# only)
                \begin{itemize}
                    \item \textbf{C\#} - Similar to null-coalesce, the advanced version is shorter and declares no local variables, instead using only the stack. This may again be an example of contextual optimisation.
                \end{itemize}
                \item Unary Increment Operators
                \begin{itemize}
                    \item \textbf{C\#} - Neither the plain nor advanced versions are mapped one-to-one in IL. The plain version uses two local variables for \codeword{y} and \codeword{z}, and \codeword{x} is maintained on the stack. In the advanced version, \codeword{x} and \codeword{y} are stored locally and \codeword{z} is maintained on the stack. Slightly different optimisation paths have been taken here producing slightly different code paths to the same result, the advanced version being two instructions longer. If all three variables used in this method were used more elsewhere, it is likely that all three would be stored in a local variable rather than one residing only on the stack.
                    \item \textbf{Java} - The plain IL is a one-to-one mapping of the source code with no optimisation visible. However, in the advanced IL makes use of a dedicated increment instruction, \codeword{iinc}, that directly maps to the \codeword{++} operator in the source code. This reduces the length of the IL by four instructions due to not needing to store the value back in the local variable as frequently.
                \end{itemize}
                \item Compound Assignment Operators
                \begin{itemize}
                    \item \textbf{C\#} - No local variable is used, the value to print is maintained only on the stack for both advanced and plain, there is no difference between the add operation in the IL.
                    \item \textbf{Java} - Unlike in C\#, no optimisation is present for the plain IL, as it maps directly to the source code. In the advanced IL however, \codeword{iinc} is used to increment the value of a local variable by three (an argument to the IL instruction), cutting the instruction count by three.
                \end{itemize}
                \item Foreach Loops
                \begin{itemize}
                    \item \textbf{C\#} - The plain IL is again effectively a one-to-one mapping of the for loop, albeit with the ordering of the three statements in the for loop signature moved to where the are actually executed. The advanced IL is eight lines longer, but both versions use the same number of external method calls within the loop itself. The advanced IL also requires the try-finally and an additional method call to dispose of the enumerator used to iterate over the list is properly after the looping is complete.
                    \item \textbf{Java} - Both versions are quite similar, sharing most normal instructions, but different external methods are called albeit that they are performing the same function (condition: \codeword{size()} vs. \codeword{hasNext()}, get element: \codeword{get()} vs. \codeword{next()}). The advanced IL does have an additional method call to get the iterator itself. Both are the same length in-place, but as with the lambdas expression, this generated more code elsewhere, namely in the constant pool, adding 35 entries, compared to the regular for loop adding only 28.
                \end{itemize}
            \end{itemize}
\newpage
\section{Conclusion}
\label{sec:conclusion}
    []
    \subsection{Review of Aims}
        The following list repeats the aims presented at the start of this report, with each aim followed by an overview of relevant results obtained.
        \begin{itemize}
            \item \textcolor{gray}{\textit{[]}}
                []
            \item \textcolor{gray}{\textit{[]}}
                []
            \item \textcolor{gray}{\textit{[]}}
                []
            \item \textcolor{gray}{\textit{[]}}
                []
            \item \textcolor{gray}{\textit{}}
                []
        \end{itemize}
    \subsection{Reflections}
        []
        \begin{itemize}
            \item survey: what does "complexity" mean? Computational? That'd be covered by slow/faster?
            \item GitHub repo selection: mistakenly thought it was `top of all repos for lang., it was not. (Check Tracy emails for reference and remedy, time permitting)'
            \item Use/Not Use, discrepancy in tense used in presented options, potentially created a bias toward `use', although this would have bene introduced AFTER the `use' question (participants can step backward through survey however)
        \end{itemize}

        \subsubsection{Lambdas}
        \label{subsubsec:lambdas}

        
    \subsection{Negative Impacting Circumstances}
        The on-going coronavirus pandemic is still very much a factor in the lives us all. :(
    \subsection{Future Research}
        []
    \subsection{Closing Statement}
        []

\newpage
\bibliography{ref}
\newpage
\section*{Appendix}
    \subsection*{A: Project Proposal}
        \includepdf[pages=-]{../Ref/proposal}
    \subsection*{B: Survey}
    \label{apx:survey}
        \includepdf[pages=-]{../Ref/survey}
\end{document}