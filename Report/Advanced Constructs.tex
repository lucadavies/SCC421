\documentclass{article}
\usepackage[utf8]{inputenc}
\usepackage[a4paper, margin=1in]{geometry}
\usepackage[font={small,it}]{caption}
\usepackage{graphicx}
\usepackage{xcolor}
\usepackage{pdfpages}
\usepackage{natbib}
\usepackage{graphicx}
\usepackage{float}
\usepackage{url}
\usepackage{nameref}
\graphicspath{{../}}
\bibliographystyle{agsm}
\hyphenation{Vonnegut}

\title{\textbf{SCC.421 Dissertation Proposal: \\ Examining Changes to ``Good'' Code Style Over Time}}
\author{
Luca Davies \\ M.Sci. (Hons.) Computer Science (with Industrial Experience)}
\date{13th March 2021}

\begin{document}
\maketitle

\newpage
\section*{Declaration}
    I certify that the material contained in this dissertation is my own work and does not contain unreferenced or unacknowledged material. I also warrant that the above statement applies to the implementation of the project and all associated documentation. Regarding the electronically submitted work, I consent to this being stored electronically and copied for assessment purposes, including the School’s use of plagiarism detection systems in order to check the integrity of assessed work. \\
    I agree to my dissertation being placed in the public domain, with my name explicitly included as the author of the work. \\
    
    \noindent
    Name: Luca Davies\\
    Date: 01/06/2021
\newpage
\section*{Abstract}
    The rich and varied syntax available to developers in most modern high-level programming languages allows for multiple differing ways to carry out any given task. As time has languages have become more developed themselves more advanced syntax constructs have been added alongside many "quality of life" type structures to make the writing code not only more efficient but also easier for developers. This report examines a number of these advanced constructs, drawing together both data from surveys taken by professional developers and data from analysis of a small number of large, active, open source code bases. Using a short but broad survey, data was collected on developers usage of these constructs and they provided information on how they felt these constructs affected their code and how often they used them, and how they thought they existed within the context of the wider industry. Analysis of code was carried using a combination of both simple regular expressions and via the user of lightweight tool to aid in the bulk processing of a large volume of source code file. It was found that ... don't know yet. FINISH ABSTRACT.
    \newline
    \newline
    [Is there any working docs?]
\newpage
\tableofcontents
\newpage

% https://www.joelonsoftware.com/2005/05/11/making-wrong-code-look-wrong/

\section{Introduction}
    \subsection{Overview}
    \subsection{Aims \& Objectives}
        The aims of this report are as such:
        \begin{itemize}
            \item ha
            \item aha
            \item ahaha
            \item ahahaha
            \item ahahahaha
        \end{itemize}
    \subsection{Report Structure}
        The remainder of this report will discuss ...
\newpage
\section{Background}
\label{sec:background)}
\section{Survey}
        []
\section{Real-world Code Analysis}
    []
\section{Findings}
    []
\section{Conclusion}
\label{sec:conclusion}
    []
    \subsection{Review of Aims}
        The following list repeats the aims presented at the start of this report, with each aim followed by an overview of relevant results obtained.
        \begin{itemize}
            \item \textcolor{gray}{\textit{[]}}
                []
            \item \textcolor{gray}{\textit{[]}}
                []
            \item \textcolor{gray}{\textit{[]}}
                []
            \item \textcolor{gray}{\textit{[]}}
                []
            \item \textcolor{gray}{\textit{}}
                []
        \end{itemize}
    \subsection{Reflections}
        []
    \subsection{Negative Impacting Circumstances}
        The on-going coronavirus pandemic is still very much a factor in the lives us all. :(
    \subsection{Future Research}
        []
    \subsection{Closing Statement}
        []
\bibliography{ref}
\newpage
\section*{Appendix}
    \subsection*{Appendix 1: []}
        []
    \subsection*{Appendix 2: []}
        []
\end{document}
